\documentclass{article}
\usepackage{graphicx} % Required for inserting images
\usepackage[utf8]{inputenc}
\usepackage[greek,english]{babel}

\title{Πρόοδος πτυχιακής εργασίας}
\author{Θεοφάνης Μπακλώρης t8210094}
\date{Μάιος 2025}

\begin{document}
\selectlanguage{greek}

\maketitle

\section{Θέμα πτυχιακής εργασίας}

Το θέμα με το οποίο θα ασχοληθώ στην πτυχιακή μου εργασία σχετίζεται με τον εντοπισμό ανωμαλιών στην καταμέτρηση ανθρώπων από χρήση αισθητήρων υπερήχων και χρήση μοντέλων μηχανικής μάθησης για πρόβλεψη του πραγματικού αριθμούν περαστικών.
Αυτό το θέμα αποτελείται από δύο σκέλη και θα προσπαθήσω να ολοκληρώσω και τα δύο ώστε να τα συμπεριλάβω στην πτυχιακή μου εργασία.

\section{Πρόοδος πρώτου μήνα και εβδομάδας 11/05/2025 - 17/05/2025}

Το πρώτο βήμα για την πτυχιακή εργασία ήταν η εύρεση του θέματος, και ύστερα την έγκριση του καθηγητή η εμβάθυνση στο αντικείμενο με το οποίο θα ασχοληθώ κατά την εκνόπηση της εργασίας. Αυτό σημαίνει ανάλυση και κατανόηση του προβλήματος, καθώς και αναζήτηση για τρόπους που ήδη υπάρχουν για την αντιμετώπισή του. Επιπλέον έρευνα στην βιβλιογραφία σχετικά με το θέμα ή επιστημονικά άρθρα με αντίστοιχο αντικείμενο. Μερικές από τις βιβλιιογραφικές αναφορές είναι οι παρακάτω:
\cite{xie2023ir}
\cite{cokbas2020}
\cite{ir-uwm}
\cite{robust}

Επίσης αναμένω την ανάκτηση των δεδομένων από την εταιρεία στην οποία κάνω την πρακτική μου άσκηση (PublicNext), μιας και η ίδια αντιμετωπίζει το ίδιο πρόβλημα, δηλαδή την μη ακριβή καταμέτρηση των επισκεπτών στα καταστήματά τους όταν δύο ή περισσότερα άτομα εισέρχονται ταυτόχρονα. Αυτό συμβαίνει λόγο του ότι οι αισθητήρες υπερήχων ,τους οποίους χρησιμοποιούν για την καταμέτρηση των επισκεπτων, δεν έχουν την δυνατότητα να καταλάβουν το πλήθος των αντικειμένων που παιρνούν από την είσοδο. 

\selectlanguage{english}
\bibliographystyle{plain}
\bibliography{references}


\end{document}
